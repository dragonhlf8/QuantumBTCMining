% !TEX TS-program = pdflatex
% !TEX encoding = UTF-8 Unicode

% This is a simple template for a LaTeX document using the "article" class.
% See "book", "report", "letter" for other types of document.

\documentclass[11pt]{article} % use larger type; default would be 10pt

\usepackage[utf8]{inputenc} % set input encoding (not needed with XeLaTeX)
\usepackage{amsmath}
\usepackage{graphicx}
\graphicspath{ {./} }
%%% Examples of Article customizations\textbf{}
% These packages are optional, depending whether you want the features they provide.
% See the LaTeX Companion or other references for full information.

%%% PAGE DIMENSIONS
\usepackage{geometry} % to change the page dimensions
\geometry{a4paper} % or letterpaper (US) or a5paper or....
% \geometry{margin=2in} % for example, change the margins to 2 inches all round
% \geometry{landscape} % set up the page for landscape
%   read geometry.pdf for detailed page layout information

\usepackage{graphicx} % support the \includegraphics command and options

% \usepackage[parfill]{parskip} % Activate to begin paragraphs with an empty line rather than an indent

%%% PACKAGES
\usepackage{booktabs} % for much better looking tables
\usepackage{array} % for better arrays (eg matrices) in maths
\usepackage{paralist} % very flexible & customisable lists (eg. enumerate/itemize, etc.)
\usepackage{verbatim} % adds environment for commenting out blocks of text & for better verbatim
\usepackage{subfig} % make it possible to include more than one captioned figure/table in a single float
% These packages are all incorporated in the memoir class to one degree or another...

%%% HEADERS & FOOTERS
\usepackage{fancyhdr} % This should be set AFTER setting up the page geometry
\pagestyle{fancy} % options: empty , plain , fancy
\renewcommand{\headrulewidth}{0pt} % customise the layout...
\lhead{}\chead{}\rhead{}
\lfoot{}\cfoot{\thepage}\rfoot{}

%%% SECTION TITLE APPEARANCE
\usepackage{sectsty}
\allsectionsfont{\sffamily\mdseries\upshape} % (See the fntguide.pdf for font help)
% (This matches ConTeXt defaults)

%%% ToC (table of contents) APPEARANCE
\usepackage[nottoc,notlof,notlot]{tocbibind} % Put the bibliography in the ToC
\usepackage[titles,subfigure]{tocloft} % Alter the style of the Table of Contents
\renewcommand{\cftsecfont}{\rmfamily\mdseries\upshape}
\renewcommand{\cftsecpagefont}{\rmfamily\mdseries\upshape} % No bold!

%%% END Article customizations

%%% The "real" document content comes below...

\title{Mining Bitcoin Using Grover's Algorithm}


\author{Shazil Arif}
%\date{} % Activate to display a given date or no date (if empty),
         % otherwise the current date is printed 

\begin{document}
\maketitle

\tableofcontents

\section{Introduction}{}

In this paper we cover the basics of Bitcoin, Blockchain and Bitcoin's Proof-of-Work Consensus Algorithm and explore what mining is. We will discuss why mining is difficult and the existing solutions that exist for it. Then we will look at how we can attempt to solve it (at a very small scale) using Grover's Algorithm.\\


\noindent \textbf{Disclaimer:} I'm not an expert in any of the topics discussed in this paper.

\section{Blockchain}{}

A blockchain is simply a ledger. It records transactions. It records the sender, receiver and the amount being sent by the sender to the receiver. It's a concept that has been around for thousands of years.\\

\noindent In ancient times, in places like Babylon and Mesopotamia, people used clay tablets and carved details of transactions onto these tablets. Slowly these evolved into papyrus documents. In the 15h century, italian mathematician Luca Pacioli introduced the idea of double entry bookkeeping. It became the underlying principle of today's field of accounting. It's very simple, each transaction involves a credit entry for one party and a debit entry for another party.\\

\noindent Blockchain is essentially a modern versional of a ledger. It is digital but also \textbf{decentralized}. We'll discuss the concept of decentralization more in depth but, at a high level what it means is that nobody owns or controls the ledger. In ancient times, ledgers were stored in temples, then banks in more modern times. Instead, the blockchain is stored on various computers referred to as a 'network'.



\section{Currency and Cryptocurrencies}{}

\section{Bitcoin}{}
\subsection{Overview}{}
\subsection{Price Factors}{}
\subsection{Byzantine General's Problem and Consenus}{}
\subsection{The Bitcoin Network and distributed Nodes}{}
\subsection{Cryptography and security}{}
\subsection{Public Adoption}{}
\subsection{Environmental Impact}{}


\section{Proof-of-Work and Mining}{}
\subsection{A brief history of Proof of Work}
\subsection{Overview}
\subsection{Hashing and the search for a nonce}
\subsection{Mining Technology}


\section{Grover's Algorithm}{}


\section{An Oracle for mining Bitcoin Blocks}{}

\subsection{Functions as Unitary Matrices}{}
\subsection{Reversible Functions}{}
\subsection{Pseudocode}{}

\section{Full Mining Implementation}{}

\section{Implementation Analysis and Practical Considerations}{}



\end{document}
